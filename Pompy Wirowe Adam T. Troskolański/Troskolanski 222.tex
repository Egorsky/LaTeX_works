\documentclass[a4paper]{book}
%\usepackage{xspace}
\usepackage[T1]{fontenc}
\usepackage[utf8]{inputenc}
\usepackage{geometry}
\usepackage{amsmath}
\usepackage{latexsym}
\usepackage{amssymb}
\usepackage[polish]{babel}
\usepackage[explicit]{titlesec}
\usepackage[normalem]{ulem}
\usepackage{lipsum}
\usepackage{enumitem}
\usepackage{textcomp}
\usepackage{setspace}
\usepackage{moj}
\usepackage{sectsty}
\usepackage{titlesec}
\usepackage{selectp}
%\usepackage{fancyhdr}
%\pagestyle{fancy}
\DeclareMathOperator{\tg}{tg}
\DeclareMathOperator{\ctg}{ctg}
\renewcommand{\baselinestretch}{0.95}
\setcounter{page}{220}
\setlength\parindent{24pt}
%\allsectionsfont{\centering}
\setcounter{chapter}{8}
%\fancyhf{}% Clear header/footer
%\fancyhead[C]{Header}
%\renewcommand{\headrulewidth}{0.4pt}
\outputonly{222,223}
\begin{document}
\pagestyle{headings}
\begin{center}
%\section*{\large Rozdział IX. Pompy helikoidalne i diagonalne}
\renewcommand{\thechapter}{\Roman{chapter}}
\chapter{Pompy helikoidalne i diagonalne}
\end{center}
\newpage
%		
	Obliczamy lub zak{\l}adamy dane potrzebne do wyznaczenia prędkości unoszenia $u_2$, a mianowicie:
\abovedisplayskip1pt
\abovedisplayshortskip1pt
\belowdisplayskip1pt
\belowdisplayshortskip1pt
\begin{enumerate}[itemsep=0.1mm]
	\item[1)] sprawność hydrauliczną wyznaczamy z wzory \textit{Łomakina} [II, 60b]
\abovedisplayskip1pt
\abovedisplayshortskip1pt
\belowdisplayskip1pt
\belowdisplayshortskip1pt
\begin{gather*}
%
1 - \eta_h = \dfrac{0{,}42}{(\lg{d_{1red}} - 0{,}172)^2}
%
\end{gather*}
\abovedisplayskip0pt
\abovedisplayshortskip0pt
\belowdisplayskip0pt
\belowdisplayshortskip0pt
Zredukowana średnica wlotu
\begin{gather*}
%
d_{1red} = 4 \cdot 10^3 \sqrt[3]{\strut \dfrac{Q}{n}} = 4\cdot10^3 \sqrt[3]{\strut \dfrac{0{,}15}{1470}}\approx 185 \: \mathrm{mm}
%
\end{gather*}
\abovedisplayskip1pt
\abovedisplayshortskip1pt
\belowdisplayskip1pt
\belowdisplayshortskip1pt
stąd
\abovedisplayskip1pt
\abovedisplayshortskip1pt
\belowdisplayskip1pt
\belowdisplayshortskip1pt
\begin{gather*}
%
1 - \eta_h = \dfrac{0{,}42}{(\lg{185} - 0{,}172)^2} = 1 - 0{,}96 \qquad \eta_h = 0{,}904 \approx  0{,}90
%
\end{gather*}
\abovedisplayskip1pt
\abovedisplayshortskip1pt
\belowdisplayskip1pt
\belowdisplayshortskip1pt
Teoretyczna wysokość podnoszenia
\abovedisplayskip1pt
\abovedisplayshortskip1pt
\belowdisplayskip1pt
\belowdisplayshortskip1pt
\begin{gather*}
%
H_{th} = \dfrac{13,7}{0,90} = 15{,}25 \: \text{m}
%
\end{gather*}
\abovedisplayskip1pt
\abovedisplayshortskip1pt
\belowdisplayskip1pt
\belowdisplayshortskip1pt
%
	\item[2)] $c_{m2} = K_{cm2}\sqrt{\smash[b]{2gH}}$, $n_{sQ} = 78{,}2$ współczynnik $K_{cm2} = 0{,}22$ (rys. VI/3);  wobec czego \\
$c_{m2} = 0,22\sqrt{\smash[b]{2\cdot9{,}81\cdot13{,}7}} = 3{,}61 \, \mathrm{m/s}$;
	\item[3)] zakładamy dla środkowej linii prądu  $\sphericalangle$ 22$^{\circ}$00$^\prime$;
	\item[4)] poprawkę \textit{Pfleiderera} przynajmujemy $1 + p = 1{,}4$;
	\item[5)] zakładamy, że woda dopływa do wirnika bez zawirowania, tj. $c_{u1}u_1 = 0.$
\end{enumerate}
\abovedisplayskip0pt
\abovedisplayshortskip0pt
\belowdisplayskip0pt
\belowdisplayshortskip0pt
\par Z wzoru [IV,14] prędkość unoszenia
\abovedisplayskip1pt
\abovedisplayshortskip1pt
\belowdisplayskip1pt
\belowdisplayshortskip1pt
\begin{align*}
%
u_2 &= \dfrac{3{,}61}{2\tg(22^{\circ})} + \sqrt{\strut \left(\dfrac{3{,}61}{2\tg(22^{\circ})}\right)^2 + \ 9{,}81 \cdot 15{,}25 \cdot 1{,}4} = 4{,}57 + {} \\ & \quad + \sqrt{\smash[b]{19{,}5 + 209}} = 4{,}57 + 15{,}15 = 19{,}62 \:  \mathrm{m/s,} \quad \text{skąd} \\
%		
%
 d_2 &= \dfrac{60\cdot19,62}{1470\cdot3,14} \approx 0{,}225 \: \text{m}. 
\end{align*}
%
\textit{Szerokość wirnika $b_2$}  
\abovedisplayskip1pt
\abovedisplayshortskip1pt
\belowdisplayskip1pt
\belowdisplayshortskip1pt
\par Zakładamy współczynnik przesłonięcia wyloty $\phi_2 = 1{,}1$; zatem przekrój	wylotowy 
\abovedisplayskip1.2pt
\abovedisplayshortskip1.2pt
\belowdisplayskip1.2pt
\belowdisplayshortskip1.2pt
\par $F_2^\prime = \dfrac{0{,}158}{3{,}61}\cdot1{,}1 = 0{,}0482 \, \mathrm{m^2}$. Szerokość $b_2 = \dfrac{0{,}0482}{\pi \cdot 0{,}256} = 0{,}061 \approx 62 \, \mathrm{mm}$.
\abovedisplayskip1pt
\abovedisplayshortskip1pt
\belowdisplayskip1pt
\belowdisplayshortskip1pt
	Obecnie możemy zaprojektować profil wirnika, wyznaczyć linie prądu i sprawdzić, czy przyjęte założena będą sluszne.

	Dla środkowej strugi $A_1A_2$ (rys. 4a) średnica $d_{A1} = 0,178 \: \text{m}$, a prędkość obwodowa \\
\abovedisplayskip1pt
\abovedisplayshortskip1pt
\belowdisplayskip1pt
\belowdisplayshortskip1pt
	$\smallskip u_{A1} = \dfrac{\pi \cdot 0{,}178 \cdot 1470}{60} = 13{,}60$ m/s; $\tg(\beta_{A1}) = \dfrac{C_{m1}}{u_{A1}} = \dfrac{4{,}33}{13{,}60} = 0,319$ i $\sphericalangle \beta_{A1} = 17$$^{\circ}$42$^\prime.$
	\newline \par Przyjmujemy kąt natarcia $\delta_1$ = 3$^{\circ}$00$^\prime$, tak iż $\sphericalangle$ $\beta_{A1}$ = 17$^{\circ}$42$^\prime$ + 3$^{\circ}$00$^\prime$ $\approx$ 20$^{\circ}$45$^\prime$
\abovedisplayskip1pt
\abovedisplayshortskip1pt
\belowdisplayskip1pt
\belowdisplayshortskip1pt	
\par Ze względu na stosunkowo wysoką wartość $n_{sQ}$ wykonamy wirnik otwarty; wobec tego ze 
względu na wytrzymałość -- należy dać łopatki grubsze niż dla wirnika zamkniętego. Przyjmujemy
4 łopatki o grubośći $s = 7$ mm na krawędzi wlotowej. Wirnik będzie wykonany z żeliwa.

Sprawdzimy obecnie wg wzoru [VII/4] zacieśnienie wlotu przez łopatki:
\abovedisplayskip1pt
\abovedisplayshortskip1pt
\belowdisplayskip1pt
\belowdisplayshortskip1pt
\begin{gather*}
%
\dfrac{1}{\phi_{1}^\prime} = 1 - \dfrac{s_1}{t_1}\sqrt{\strut 1 + \dfrac{\ctg^2(\beta_{1}^\prime)}{\sin^2(\lambda_1)}}; \qquad t1_1 = \dfrac{178\pi}{4} = 140 \: \text{mm} \qquad \sphericalangle \beta_{1}^\prime = 20^{\circ}45^\prime.
%
\end{gather*}
\abovedisplayskip1pt
\abovedisplayshortskip1pt
\belowdisplayskip1pt
\belowdisplayshortskip1pt
\par Z wstępnego szkicu wirnika $\sphericalangle$ $\lambda_1$ $\approx$ 58$^{\circ}$00$^\prime.$ Po podstawieniu znajdziemy
\abovedisplayskip1pt
\abovedisplayshortskip1pt
\belowdisplayskip1pt
\belowdisplayshortskip1pt
\begin{gather*}
%
\dfrac{1}{\phi_1^\prime} = 1 - \dfrac{7}{140}\sqrt{\strut 1 + \dfrac{2{,}64^2}{0{,}848^2}} = 0{,}836, \qquad \text{skąd} \enskip \phi_1^\prime = \dfrac{1}{0{,}836} \approx 1{,}2
%
\end{gather*}
\abovedisplayskip0pt
\abovedisplayshortskip0pt
\belowdisplayskip0pt
\belowdisplayshortskip0pt
tj. zgodnie z założeniem.	
\newpage

%\paragraph{\textsection 1. Pompy helikoidalne}\\  
%
\textbf{Obliczenie kąta nachylenia łopatek na wlocie $\beta_1$} \\
%
Wirnik dzielimy na 4 elementarne wirniki o jednakowej wydajności, pry czym prędkość połudikową $c_{m1}$ obliczona dla środkowej linii prądu przyjmujemy stałą dla wszystkich linij prądu, po uwzdlędnieniu kąta natarcia $\delta_1$, tj.
\abovedisplayskip0pt
\abovedisplayshortskip0pt
\belowdisplayskip0pt
\belowdisplayshortskip0pt
\begin{gather*}
%
c_{m1} = u_{A1}\tg(\beta_1^\prime + \delta_1) = 13{,}6\tg(20^{\circ}45^\prime) = 5{,}15 \: \text{m/s}.
%
\end{gather*}
\abovedisplayskip0pt
\abovedisplayshortskip0pt
\belowdisplayskip0pt
\belowdisplayshortskip0pt
Obliczenie przeprowadzamy tabelarycznie. 
\vspace*{-\baselineskip}
\begin{table}[h!]
	\begin{center}
	\begin{tabular}{c|c|c|c|c|c|c}
		\hline
		\multicolumn{1}{l|}{\begin{tabular}[c]{@{}l@{}}Linia \\ prądu\end{tabular}} & \begin{tabular}[c]{@{}c@{}}Średnica\\ $d_1$\\ m\end{tabular} & \begin{tabular}[c]{@{}c@{}}$u_1$\\ m/s\end{tabular} & \begin{tabular}[c]{@{}c@{}}$c_{m1}^\prime$\\ m/s\end{tabular} & \begin{tabular}[c]{@{}c@{}}$\tg(\beta_1)$ = \\ = $c_m/u_1$\end{tabular} & $\sphericalangle$$\beta_1^\prime$ & \multicolumn{1}{l}{$w_1$ = $\dfrac{c_{m1}^\prime}{\mathrm{\sin(\beta_1^\prime)}}$} \\ \hline 
		$B_1B_2$ \rule{0mm}{6mm}                             & 0,245                                                     & 18,75                                            & 5,15                                               & 0,275                                                     & 15$^{\circ}$20$^\prime$        & 19,50                                 \\ 
		$C_1C_2$                              & 0,215                                                     & 16,45                                            & 5,15                                               & 0,313                                                     & 17$^{\circ}$23$^\prime$        & 17,25                                 \\ 
		$A_1A_2$                              & 0,178                                                     & 13,60                                            & 5,15                                               & 0,380                                                     & 20$^{\circ}$45$^\prime$        & 14,60                                 \\ 
		$D_1D_2$                             & 0,134                                                     & 10,25                                            & 5,15                                               & 0,503                                                     & 26$^{\circ}$43$^\prime$        & 11,47                                 \\ 
		$E_1E_2$                               & 0,070                                                     & \phantom{1}5,35                                             & 5,15                                               & 0,963                                                     & 43$^{\circ}$55$^\prime$        & 7,43                                 \\[6pt]  \hline
	\end{tabular}
	\end{center}
\end{table}
\vspace*{-\baselineskip}
\abovedisplayskip0pt
\abovedisplayshortskip0pt
\belowdisplayskip0pt
\belowdisplayshortskip0pt
\par\textbf{Obliczenie kątów $\beta_2$ nachylenia łopatek na wylocie} \\
\vspace*{-\baselineskip}
\par Momenty statyczne poszczególnych linij prądu są następujące: $M_{st}(B_1B_2)$ = 0,0115 $\text{m}^2$; $M_{st}(C_1C_2)$ = 0,0105 $\text{m}^2$; $M_{st}(A_1A_2)$ = 0,0113 $\text{m}^2$; $M_{st}(D_1D_2)$ = 0,0105 $\text{m}^2$; $M_{st}(E_1E_2)$ = 0,0095 $\text{m}^2$.
\abovedisplayskip1pt
\abovedisplayshortskip1pt
\belowdisplayskip1pt
\belowdisplayshortskip1pt
\par Sprawdzimy przyjętą liczbę łopatek wg wzoru \\
$Z \approx$ 13 $\dfrac{r_m}{e}\text{sin}\left(\dfrac{\beta_1 + \beta_2}{2}\right);$ dla środkowej linii prądu $r_m$ = 100 mm, $e$ = 112 mm, $\sphericalangle \beta_1^\prime$ = 20$^{\circ}$45$^\prime$,\\
$\sphericalangle \beta_1^\prime$ = 22$^{\circ}$00$^\prime$. Po podstawieniu znajdziemy $Z$ = 13$\dfrac{100}{112}\text{sin}\left(\dfrac{22^{\circ}00^\prime + 22^{\circ}00^\prime}{2}\right) \approx$ 4,2 -- bliskie założonej wartości $Z$ = 4; nie byłoby jednak błędem przyjęcie $Z$ = 5. \\
\vspace*{-\baselineskip}
\par	Sprawdzimy presłonięcie wloty wirnika przez łopatki dla środkowej linii prądu $A_1A_2$; \\
\abovedisplayskip1pt
\abovedisplayshortskip1pt
\belowdisplayskip1pt
\belowdisplayshortskip1pt
\begin{gather*}
%
t_2 = \dfrac{255\pi}{4} \approx 200 \: \text{mm}; \enskip \text{grubość łopatki} \enskip s = 7 \: \text{mm};
%
\end{gather*}
\abovedisplayskip1pt
\abovedisplayshortskip1pt
\belowdisplayskip1pt
\belowdisplayshortskip1pt	
\begin{gather*}
%
s_{u2} = \dfrac{7}{\mathrm{\sin(22^{\circ})}} = 18{,}7 \: \text{mm} \quad \phi_2 = \dfrac{t_2}{t_2 - s_{u2}} = \dfrac{200}{200 - 18{,}7} = 1{,}103,
%
\end{gather*}
bliskie założonej wartości $\phi_2$ = 1,1. Przyjmujemy $\phi_2$ = 1,103 dla wszystkich linij prądu. 
\par Według zaprojektowanego profilu wirnika obliczamy jego prekrój wylotowy, jako sumę przekrojów wirników elementarnych:
\abovedisplayskip1pt
\abovedisplayshortskip1pt
\belowdisplayskip1pt
\belowdisplayshortskip1pt
\begin{gather*}
%
F_2 = 2 \pi (1{,}4 \cdot 13{,}6 + 1{,}55 \cdot 13{,}05 + 1{,}67 \cdot 12{,}5 + 1{,}8 \cdot 11{,}8) = 510{,}7 \enskip \mathrm{cm^2} = 0{,}05107 \enskip \mathrm{m^2}
%
\end{gather*}
\abovedisplayskip0pt
\abovedisplayshortskip0pt
\belowdisplayskip0pt
\belowdisplayshortskip0pt
\par Swobodny przekrój wylotowy $F_2^\prime$ = $\dfrac{0{,}05107}{1{,}103}$ = 0,0463 $\mathrm{m^2}$. Prędkość merydionalna u wylotu $c_{m2}$ = $\dfrac{0{,}158}{0{,}0463}$ = 3,42 m/s zamiast wartości $c_{m2}$ = 3,61 m/2 obliczonej wg wartości współczynnika $K_{cm2}$. Różnica w prędkości wynosi zaledwie 3,0\%, wobec czego nie przeprowazamy korekty profilu wirnika. 
\abovedisplayskip0pt
\abovedisplayshortskip0pt
\belowdisplayskip0pt
\belowdisplayshortskip0pt
\par	Prędkość $c_{m2}$ = 3,42 m/s przyjmujemy jako stałą wzdłuż wylotowej krawędzi łopatki; krzywizna bowiem linij prądu jest łagodna.
	
	\textit{Poprawkę Pfleiderera} dla każdej linii prądu obliczymy z wzory [III, 62] $p$ = $\dfrac{r_z^\prime \psi^\prime}{z M_{st}}$, gdzie $\psi^\prime = (1{,}0\div1{,}2)(1+\sin(\beta_2))r_1/r_2$.
	\abovedisplayskip1.2pt
	\abovedisplayshortskip1.2pt
	\belowdisplayskip1.2pt
	\belowdisplayshortskip1.2pt
	\par Np. dla środkowej linii prądu $A_1A_2$:
\begin{gather*}
%
\psi^\prime = (1{,}0\div1{,}2)(1 + \sin(22^{\circ}))\dfrac{89}{128} = 0{,}96 \div 1{,}15.
%
\end{gather*}
	
\end{document}